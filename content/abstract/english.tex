This bachellor thesis presents the development of a graphics engine written in Common Lisp,
leveraging the capabilities of OpenGL.
The primary goal of the project was to design and build a functional,
user-friendly engine to render 3D scenes.
The project was personally motivated by the desire to have a flexible and direct tool for graphics programming without the need to learn a complex framework or editor.

The engine provides a simple-to-use interface for loading and rendering meshes from files,
generating procedural meshes using Perlin noise,
mapping textures to meshes,
and simulating lighting.

Despite the challenges inherent in graphics programming from scratch,
the project was successful in producing a working graphics engine,
adhering closely to the initial scope and achieving the planned feature set.

This project thus makes a significant contribution to the field of graphics programming,
especially within the Common Lisp community,
by providing a tool that not only meets the project's goals but also offers an excellent platform for further expansion and enhancement.
