Este trabajo de fin de grado presenta el desarrollo de un motor gráfico programado en Common Lisp,
haciendo uso de las capacidades de OpenGL.
El objetivo principal del proyecto era diseñar e implementar un software funcional y sencillo de usar,
para renderizar escenas 3D.
Estuvo motivado por un deseo personal de tener una herramienta versátil y directa para la programación gráfica,
sin tener que aprender a utilizar un editor o framework complejo.

El motor presenta una interfaz sencilla de utilizar para cargar y renderizar mallas de ficheros,
generar mallas procedurales con ruido Perlin,
enlazar texturas a mallas,
y simular iluminación.

A pesar de los desafíos que existen en la programación gráfica,
el proyecto fue exitoso a la hora de producir un motor gráfico funcional.
Se mantuvo dentro del umbral definido y se implementaron todos los requisitos que se especificaron.

En conclusión,
el proyecto contribuye significativamente al campo de la programación gráfica,
especificamente dentro de la comunidad de Common Lisp.
Ofrece una herramienta que además de cumplir con sus objetivos,
presenta una plataforma excelente para futuras mejoras o expansiones.
