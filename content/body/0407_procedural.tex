% Talk about marching cubes

\label{SEC:PROCEDURAL}

\index{Mesh!Procedural Mesh}

Procedural mesh generation allows creating dynamic 3D shapes algorithmically.
Instead of relying on static models,
it facilitates the development of complex objects whose form can change based on predefined or random variables.
This in turn enables the creation of organic and natural-looking environments.
In this engine,
the process is broken up into two steps.

\subsection{Base mesh}

When generating procedural meshes,
it is very useful to have a starting point.
I found the easiest starting point was a simple plane.
Multiple planes can be combined to form cubes,
and the vertices of a cube can be normalized to form a sphere.

OpenGL supports a type of indexed mesh called a triangle strip.
In this mesh,
the \ac{vbo} as always holds all the unique vertices.
The \ac{ebo} however,
instead of indexing individual triangles,
will make use of the fact that in a plane most vertices are shared by multiple triangles.

\subsection{Mesh manipulation}

Once a base mesh has been generated,
transformations can be applied to it to get procedural or even organic shapes.
If for example we sample a 2D plane of fractal noise based on perlin noise,
and use the random values to modify the height of each vertex,
we can easily create natural looking terrain.
See figure \ref{FIG:TERRAIN}

\begin{figure}[Procedural terrain mesh]{FIG:TERRAIN}{A procedural mesh generated using a plane of triangle strips, and fractal noise.}
  \subfigure[SBFIG:TERRAIN_LINES]{Rendered showing underlying geometry}{\image{0.4\textwidth}{}{terrain_lines}}
  \subfigure[SBFIG:TERRAIN_FILLED]{Rendered using triangles}{\image{0.4\textwidth}{}{terrain_filled}}
\end{figure}


% \subsection{Marching cubes}
