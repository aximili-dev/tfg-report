% Describe phong lighting model
% Implmented in shaders primarily
% Material and light structs

\subsection{Lighting}

\index{Phong Reflection Model}
\index{Rendering!Reflections}

In order to achieve lighting effects,
the Phong reflection model was implemented.
It was developed by Bui Tuong Phong in 1975\cite{phong}.
It's an empirical model of local illumination.
It describes surface reflections as the combination of diffuse reflection,
which falls off gradually,
and specular reflection,
which causes small highlights.
It also introduces a third \textit{ambient} term,
to account for light that is scattered about an entire scene.

The way it's implemented in this project is using an OpenGL shader program.
The first stage, the vertex shader,
runs for every vertex in the mesh and takes care of transforming it from world space to screen space.
The second stage, the fragment shader,
takes the transformed vertex and shades it.
It calculates the three terms of the Phong model separately,
and then just adds them together for the final result.

The final result will depend on the amount of lights that influence the mesh being lit.
Since running the calculations for every pixel,
and for every light can be very inefficient,
there are many ways to simplify this.
The easiest is to break a scene up into chunks,
this way the rendering code can ensure that a mesh is only lit by the lights that are a certain vicinity,
reducing the number of calculations that have to be made.

\subsection{Texture Mapping}

\index{Texture Mapping}

Another important feature when it comes to lighting effects is texture mapping.
By mapping a diffuse texture as well as a specular texture,
the Phong model detailed in the previous section shows its true power.
When shading flat surfaces,
it can give the impression of depth and reflections without extra geometry.
An example is shown in figure \ref{FIG:PHONG}\footnote{
\href{https://learnopengl.com/img/textures/container2.png}{Diffuse texture} and
\href{https://learnopengl.com/img/textures/container2_specular.png}{Specular map} obtained from
\href{https://learnopengl.com/Lighting/Lighting-maps}{Learn OpenGL}
\textcopyright\,
\href{https://twitter.com/JoeyDeVriez}{Joey de Vries}
\href{https://creativecommons.org/licenses/by/4.0/legalcode}{CC BY 4.0} }

% \url{https://learnopengl.com/img/textures/container2.png} and \url{https://learnopengl.com/img/textures/container2_specular.png} \textcopyright Joey de Vries (\url{https://twitter.com/JoeyDeVriez}) (\url{https://learnopengl.com/Lighting/Lighting-maps}) \href{https://creativecommons.org/licenses/by/4.0/legalcode}{CC BY 4.0}

\begin{figure}[Phong lighting with texture mapping]{FIG:PHONG}{A cube mesh with diffuse and specular textures mapped to it, shaded using the Phong Reflection Model.}
  \subfigure[SBFIG:CONTAINER_DIFFUSE]{\href{https://learnopengl.com/img/textures/container2.png}{Diffuse texture}}{\image{}{3cm}{container}}
  \subfigure[SBFIG:CONTAINER_SPECULAR]{\href{https://learnopengl.com/img/textures/container2_specular.png}{Specular map}}{\image{}{3cm}{container.specular}}
  \subfigure[SBFIG:CONTAINER_RENDER]{Rendered mesh}{\image{}{3cm}{container_render}}
\end{figure}

Other types of textures can be mapped to the meshes for even more detail.
Normal maps,
displacement maps and bump maps can create the illusion of higher detail geometry,
without actually increasing the number of vertices.
Bump maps give the impression of small raised details on a surface,
making it look less flat.
Normal maps improve upon bump maps,
by giving raised details an angle relative to the surface they're on.
Displacement maps create higher detail by transorming the underlying mesh.
These types of texture maps and their corresponding lighting calculations are outside of the scope of this project.
