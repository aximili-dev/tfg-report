Graphics engines are some of the most complex pieces of software that exist.
This is because of the many systems they tend to implement,
and these systems having to integrate well with each other.
As a consequence,
they have become an important part of a wide array of applications and industries.
Scene graphs and scripting systems allow for rapid development.
Developers can use detailed lighting and texturing models for virtual reality visualizations or architectural design.
In short,
any type of creator who wants to express their idea in some three-dimensional way,
will likely use a graphics engine.
The following is a more detailed list of the different systems that can be available in such an engine.
Some of these systems could qualify a piece of software as a game engine rather than just a graphics engine.

\begin{itemize}
\item \textbf{Resource Management.}
  In charge of loading assets from storage into memory,
  and mapping them onto the relevant data structures the engine uses.
\item \textbf{Meshes.}
  Holds references to 3D models and the GPU buffers where they are stored.
  Also allows a program to manipulate a mesh,
  so that a model can change shape or be animated.
\item \textbf{Lighting.}
  A collection of shaders that get compiled into a rendering pipeline.
  This pipeline instructs the GPU how to render a mesh.
\item \textbf{Sound.}
  An API to define different sound sources, and mix them together.
\item \textbf{Animation.}
  Handles keyframes and different mesh states to animate between them.
\item \textbf{Networking.}
  A higher level abstraction on top the operating system's network stack,
  to share game state over the network for different players.
\item \textbf{Input Handling.}
  Can be done by specifying callbacks that should be run on any input event,
  or by polling the engine for the current input state at a fixed rate.
\item \textbf{Physics.}
  Many subsystems actually comprise the physics system.
  Some of them are detailed below.
  \begin{itemize}
  \item {Rigid body physics.}
    Models meshes as rigid bodies that react to gravity and collide with each other.
  \item {Soft body physics.}
    Models meshes as malleable bodies that can deform upon collision.
  \item {Cloth simulations.}
    Model a plane as a piece of cloth that can be deformed and ripped.
  \item {Fluid simulations.}
    Model a collection of points as if they were particles in a fluid.
  \end{itemize}
\item {Navigation.}
  Offers implementations of navigation meshes and agents to easily model autonomous behavior.
\end{itemize}

This list is again by no means exhaustive,
many other features are made available by some engines.
