% Short descriptions of cl-opengl and cl-glfw
% Short description of glfw and opengl initialization and termination
% Challenges found
% - Lots of raw binary data manipulation, hard to make pretty
% - OpenGL errors, solved by creating some lisp macros and functions

The integration of OpenGL could be considered the core of the engine architecture.

\subsection{with-graphics macro}

The finished engine provides a macro called \texttt{with-graphics},
that takes care of the following steps.

\paragraph{OpenGL and GLFW initialization}

GLFW is used to create an operating system window,
and an OpenGL context to go with it.
Context version 4.5 is specified to enable modern OpenGL features like rendering and compute shaders.

\paragraph{Graphics class}

A graphics object is created to hold all the state relating to the engine.
This includes the window size,
current keyboard state,
global objects like the debug font,
whether debug mode is enabled,
etc.
This object is then exposed in the scope of the expanded macro,
to allow any program to interact with it.

\paragraph{OS Callbacks}

GLFW is used to setup some basic callbacks to integrate with the \ac{os}.
The most important is the resize callback,
which updates the window size in the graphics object.
Another important one is setting up a keyboard callback to quit when the escape key is pressed.
This can be overridden by a program using the macro,
but before this is done,
it's an easy way to quit a program during the development phase.

\subsection{Data structures and buffers}

Another small but important aspect of integrating OpenGL is how to move data between Common Lisp and OpenGL buffers.
For that purpose the function \texttt{make-gl-array} and the macro \texttt{with-gl-array} were written.
The function takes care of allocating memory for a C-style array,
and copying data from a lisp array to it.
The macro allows you to create a contained scope where that array exists,
so you can use it to move data to a buffer or do other manipulations,
and afterwards frees the memory that was allocated to it.

% \subsection{OpenGL error handling}
