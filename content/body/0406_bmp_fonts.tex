% Mention texture atlases
% Describe the math and shader used

\index{Rendering!Text}

One important but easily overlooked aspect of this project was rendering text.
Once you are rendering complex scenes,
getting real-time debug information in a console is very overwhelming.
Since usually you only want to see the current state of certain variables,
showing them on screen is the best option.

It is possible to render vector fonts like TTF or OTF formats,
but since that requires a library for the rendering,
and there aren't many to choose from for Common Lisp,
I opted for bitmap fonts instead.

A bitmap font is a font that is entirely defined in one image file.
The downside of this is that scaling the font by any fractional factor will result in blurry text,
since the pixel size is fixed.
The upside is that rendering any character comes down to locating it in a texture,
and mapping that texture to some surface.

The font I chose is licensed under Creative Commons 0,
and it has all its characters conveniently placed as a function of their Unicode code point.
I took some measurements using GIMP,
and came up with the following numbers.

\begin{itemize}
\item The width of each character is 7 pixels.
\item The height of each character is 11 pixels.
\item The X position of each character is $98 + 14C_{3,0}$, where C is the last four bits of the character's code point.
\item The Y position of each character is $66 + 13C{15,4}$, where C is the first 12 bits of the character's code point.
\end{itemize}

An example of some text rendered with this font can be seen in figure \ref{FIG:TERRAIN}
