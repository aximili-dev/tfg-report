One of the goals of this project is the ability to generate procedural meshes.
A very important tool to that end is 3D smooth noise.
When we talk about noise,
we are talking about any \textit{n} dimensional function that will output a random value.
In the case of smooth noise,
the desired effect is for those values to change gradually when the input parameters also change gradually.
One such example can be seen in figure \ref{SBFIG:PERLIN}.
In order to generate higher detail and a more organic look,
lower frequency, scaled up noise (or vice-versa) can be blended on top,
like in figure \ref{SBFIG:PERLIN_FRACT}.

\begin{figure}[Perlin noise example]{FIG:PERLIN}{2D slice through 3D Perlin noise.}
  \subfigure[SBFIG:PERLIN]{2D slice through 3D Perlin noise}{\image{0.4\textwidth}{}{perlin}}
  \subfigure[SBFIG:PERLIN_FRACT]{Layered perlin noise}{\image{0.4\textwidth}{}{perlin_fractal}}
\end{figure}

\subsection{Perlin noise}

The noise function that has been chosen for this project is Perlin noise.
It was developed by Ken Perlin in 1983 because he was frustrated with the ``machine-like'' look of imaged generated by computers at the time.\cite{making_noise}
The algorithm to generate Perlin noise has been broken down below.

\subsubsection{The algorithm}

\paragraph{Grid definition}
Define a three-dimensional grid,
where each grid point has associated with it a random gradient vector.
In order to get a uniform distribution of random vectors,
I used the sphere point picking method described by Wolfram MathWorld\cite{sphere_point}.

\paragraph{Dot product}
To determine the noise value of a candidate point,
determine the grid cell its in.
For each of the corners of that cell,
calculate the position vector of the point relative to that corner.
Take the dot product of this new vector with the gradient of the respective corner.

\paragraph{Interpolation}
Finally interpolate between all the dot products calculated in the previous step.
Interpolation if done using the smooth-step function describe in equation \ref{EQ:SMOOTH}

\begin{equation}[EQ:SMOOTH]{Smoothstep function}
\boxed{f(min, max, x) = min + (max - min)(3 - 2x)x^2}
\end{equation}
