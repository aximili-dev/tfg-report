The previous sections of this chapter give context about the current state of the tools we're using.
This next section discusses existing work at the intersection of Graphics Engines, OpenGL and Common Lisp.
In this small slice,
there are a few software packages that have to be mentioned.
However as the following sections show,
the Common Lisp Graphics landscape is almost abandoned.

\subsection{cl-opengl}

\texttt{cl-opengl} is a common lisp library that provides bindings the OpenGL graphics API.
The way it achieves this is using the \ac{cffi}.
Besides offering low level bindings directly the OpenGL \ac{api},
this library also creates its own interface on top of it.
This higher level interface is very desirable,
since it prevents us from having to call foreign functions directly,
which generally means working with dynamic and memory directly\cite{cffi}.

This library also includes bindings to the \ac{glut} \ac{api},
however in this project \texttt{cl-glfw} is used instead,
which provides bindings to \ac{glfw}.
The reason for this is that \ac{glfw} gives full control of the render loop to the developer.

\subsection{clinch}

% TODO: Find out how to properly quote
Clinch purports itself as \textit{``a simple, yet powerful 3d game engine for Lisp''}. It is written on the same \texttt{cl-opengl} library as this project, although its feature\-set is quite more extensive. The main features it offers that are not present in this engine are the following\cite{clinch}:

\begin{itemize}
\item \textbf{3D physics with joints and motors}
\item \textbf{2D vector graphics}
\item \textbf{Animation of textures and 3D objects}
\end{itemize}

The library's homepage mentions that the project is still under development,
however no changes have been uploaded since 2017 as of the writing of this text.

\subsection{cl-horde3d}

Another library acting as a wrapper by way of \ac{cffi} bindings.
In this case the underlying \ac{api} is from the Horde3D library\cite{horde3d}.
At the time of writing,
no updates have been provided for this library since 2013.
