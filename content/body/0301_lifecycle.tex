% No real methodology was used
% Closest would be iterative and incremental development
% - Initialization step: Port Learn OpenGL tutorials to lisp
% - Project control list: (more detail in requirements)
%   - BMP fonts
%   - Render meshes
%   - Phong lighting
%   - Load meshes from files
%   - Create mini-framework for easy bootstrapping
%   - Generate procedural meshes
%
% During each iteration, anything that was needed was implemented on the spot

\label{SEC:LIFECYCLE}

\index{Iterative and Interactive Development}

As the only developer of this project,
I approached the development cycle in a way that aligned with my personal style and habits.
While no official methodology was followed,
the closest approximation would be Iterative and Incremental Development.
The idea behind this method is to have a system evolve over the course of small cycles.
Even though it wasn't strictly followed,
it still makes sense to talk about this project's lifecycle using the nomenclature this method provides.

\subsection{Initialization step}

The initialization step is meant to create a base version of the software system.
If the entire project were reduced to small iteration cycles,
there would be nothing to start from.
For this project,
this step consisted of implementing parts of \textit{Learn OpenGL}\cite{learnopengl} in Common Lisp.
This allowed me to learn the basics of OpenGL,
while at the same time working towards a base state for the project.

\subsection{Project control list}

The project control list is intended to guide the iteration process,
by keeping a record of the tasks that need to be performed,
and goals that wish to be achieved.
It's made up of the items below.
Furthermore,
any features that were deemed necessary to implement one of these items were implemented on the spot during the respective cycle.

\begin{itemize}
\item \textbf{Rendering bitmap fonts.}
\item \textbf{Rendering meshes.}
\item \textbf{Phong lighting.}
\item \textbf{Loading meshes from files.}
\item \textbf{Mini-framework for easy bootstrapping.}
\item \textbf{Generating procedural terrain.}
\item \textbf{Generating procedural 3D environments.}
\end{itemize}
