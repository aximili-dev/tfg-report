Videogames are little suited to \ac{oop}.
As the complexity of a game increases,
so do the amount of properties each game element has.
When trying to achieve this through regular inheritance,
we end up with highly complex, multiple inheritance graphs.
With an entity system, every facet of a gameplay element is implemented as a component.
For example, anything affected by gravity should have a ``massive'' component.
Figure \ref{fig:oop_vs_ecs} shows a slightly more complex example comparing the two systems.

\begin{figure}
  [Object oriented programming vs entity based programming]
  {fig:oop_vs_ecs}
  {
    Difference between an inheritance based system and an entity based system.
    In the example shown in figure \ref{fig:messy_inheritance},
    making more complex game items would mean creating new types,
    whereas in figure \ref{fig:entity_diagram} new types are created dynamically by grouping components together.
  }

  \subfigure[fig:messy_inheritance]{Inheritance based design}{\image{0.4\textwidth}{}{messyinheritance.drawio}}
  \subfigure[fig:entity_diagram]{Entity based design}{\image{0.4\textwidth}{}{prettyecs.drawio}}
\end{figure}
