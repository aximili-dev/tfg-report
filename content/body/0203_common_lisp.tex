% Talk about programming in common lisp
% - Interactive development environment (short, more detail in methods and materials)
% - Homoiconicity
% - First language used for computer rendering

\label{SEC:CL}

Common Lisp exists as a dialect in a large family of Lisp programming languages.
It is important to note that Common Lisp itself is not a language implementation,
but rather a language specification.
It has been published in ANSI standard \textit{ANSI INCITS 226-1994 (S20018)}.
There is also the Common Lisp HyperSpec,
an HTML version which has been derived from the standard with permission from ANSI and X3.\cite{clhs}

There are many features that set Common Lisp apart from other programming languages.
Here are some that were particularly helpful for this project.
Note that for the rest of this document,
Common Lisp and lisp will be used interchangeably.

\subsection{REPL driven development}

The first thing that stands out from Common Lisp is that there is no real concept of ``program''.
Rather when you start the interpreter,
you are faced with a \ac{repl}.
In this environment you can execute any lisp code,
such as in listings \ref{COD:REPL_1} and \ref{COD:REPL_2}.
By default,
any new functions or data structures you define will be confined to this \ac{repl}.
There are two different ways to make your code persistent.
The first is by saving the current state of the \ac{repl} to an image,
which can be booted up at a later date to continue working with it.
The second one,
and the one used for this project,
is by using an \ac{ide} capable of linking code you write in files,
with an existing \ac{repl} session.
The benefits of the second approach are that you get to have a static copy of the code you write,
which is beneficial for source control and collaboration.
The setup used for this project will be explained in more detail in \secref{SEC:DEV_ENV}.

\LispCode[COD:REPL_1]{REPL example \textemdash Code execution}{Some basic lisp functions being executed. In this case the \texttt{+} and \texttt{nth} functions}{repl-1.lisp}{}{}{}

\LispCode[COD:REPL_2]{REPL example \textemdash Function definition}{This function definition would be lost if the proper steps aren't taken before closing the REPL session.}{repl-2.lisp}{}{}{}

One major consequence of how the Common Lisp \ac{repl} works,
is that you get code reloading for free.
What this means is that while the code is running,
you can modify a specific function,
recompile it,
and see the changes take effect without having to restart the entire program.
This is also very useful for debugging,
since whenever an error is encountered,
the lisp debugger will allow you to fix it while execution is paused.

Another way this interactive development experience turned out useful for this project,
is that rendering code could be modified while the application is running.
This isn't always helpful since certain code in the engine only gets run once,
but for code that is run every frame,
it was possible to get immediate feedback without having to restart the application.

\subsection{Homoiconicity}

Homoiconicity is a property of certain programming languages,
where the program structure can be manipulated as data in the same language.
In the context of Common Lisp,
this property emerges from the use of \acs{sexpr}.
The same notation is used to represent lisp code,
such as function invocations,
as well as the main lisp data structure,
the list.
For a more concrete example refer back to listing \ref{COD:REPL_1},
where an \ac{sexpr} is used for the \texttt{nth} function invocation,
and also to define its second argument.

One other way this manifests in Common Lisp is in its powerful macro system.
Since code and data are treated the same,
no separate macro language is needed.
This makes writing macros very easy.

\subsection{Computer Graphics history}

A final interesting point about Common Lisp is that it has a rich history in Computer Graphics.
Lisp machines build in the 1980s by companies like Symbolics were some of the first capable of real time video I/O,
and the first to produce HDTV video\cite{symbolics}.
