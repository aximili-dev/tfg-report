%\begin{itemize}
%\item Modularize the code more, separate into packages maybe even libraries
%\item Implement scene management with a scene tree
%\item Implement a UI toolkit, or integrate an existing one if there is one
%\item Implement a sound system, have sound sources, sinks and mixers
%\item Implement rigid body physics, and colliders for different geometries
%\end{itemize}

\label{SEC:RECOMMENDATIONS}

After having finished the development,
there are still many areas where improvements are possible.
There are also many features that could be implemented and integrated into the engine,
so as to have a more complete feature set.

\subsection{Code refactoring}

Due to the nature of the project's lifecycle,
most of the code has ended up in one shared package.
Before adding any new features,
there would be value in splitting the code up into multiple packages,
maybe even extracting some libraries for some specific cases,
like the Wavefront OBJ parser.
This would not only make the current code base more digestible,
but it would make it easier to expand on the work in the future.

\subsection{Scene management}

The project currently has a hierarchy where meshes belong to models,
and models belong to entities,
with each level in this hierarchy being responsible for a different part of the rendering process.
Entities are located in the scene and can contain a model,
models can have materials and textures and a mesh,
and the mesh is just a collection of vertices and indices.

A feature that would allow for more complex scenes would be a scene or world graph,
wherein entities can be parents or children to other entities.
This way different components or elements can be composed to create complex objects.
One consideration for this is that the transforms of entities have to be combined together for any operation that depends on their position, scale or rotation.

\subsection{UI Toolkit}

In the case where a graphics application has to be interactive,
a UI toolkit goes a long way towards reaching that goal.
Implementing such a toolkit in this engine would require some work,
but it would be well worth the effort.

An alternative to implementing such a toolkit from scratch,
would be integrating an existing library like Alloy.
It's written in Common Lisp and contains code to interface with OpenGL and GLFW\cite{alloy}.

%\subsection{Sound system}


\subsection{Rigid body physics}

Another interesting area for expansion in the current project would be the implementation of a rigid body physics system.
Graphics rendering is only one aspect of creating an immersive and interactive 3D environment.
Realistic physics simulations can significantly enhance the overall feel and realism of the scene.

Rigid body physics would allow for the simulation of physical interactions between objects in the scene,
including factors such as gravity,
friction,
and force-induced movement.
This would bring another layer of realism to the rendered scenes,
providing a more engaging and interactive experience for the users.

In addition to the physics system,
a collision detection and response system would be an essential addition.
Collision systems allow for the detection of interactions between objects in a scene,
such as when they touch,
overlap or collide.
Once a collision is detected,
appropriate responses such as bouncing or sliding can be simulated,
adding another level of authenticity to the scene.
