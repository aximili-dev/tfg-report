To represent 3D models in software,
and storing them in a way that can be understood by OpenGL,
many different parts were written that all work together.

\subsection{Vertex Primitives}

In order to render an object,
we need to know its visual properties.
The best way to represent this is by sampling a series of points on the object's surface.
If we group the visual properties of these points together,
we get vertices.
When talking about these properties in this context,
we call them vertex attributes\cite{gameengine_arch}.
There are many attributes that are typically included,
but in this engine we only use the following

\begin{itemize}
\item \textbf{Position vector}
  $(p_i = [\begin{matrix} p_{ix} & p_{iy} & p_{iz} \end{matrix}])$.
  The position of this vertex in 3D space.
  This is usually relative to the object's origin,
  or what is also called \textit{model space}.
\item \textbf{Normal vector}
  $(p_i = [\begin{matrix} n_{ix} & n_{iy} & n_{iz} \end{matrix}])$.
  This vector defines the surface normal at this point.
  It's used for dynamic lighting calculations.
\item \textbf{Diffuse color}
  $(d_i = [\begin{matrix} d_R & d_G & d_B \end{matrix}])$.
  This vector describes the diffuse color of the surface at this point,
  expressed in the RGB color space.
  If we were to implement blending in this project,
  we would also include an \textit{alpha} value here.
\item \textbf{Texture coordinates}
  $(t_{ij} = [\begin{matrix} u_{ij} & v{ij} \end{matrix}])$.
  This vector maps each vertex onto a texture.
  One way to visualize this is to imagine the object unwrapping,
  and then flattening onto a 2D plane.
\end{itemize}

\subsubsection{Vertex formats}

Depending on the mesh we want to render,
we will want to include some vertex attributes in the vertices while omitting others.
For example,
when rendering a detailed 3D model,
it is desirable to keep as much of that detail as possible in a texture,
and then use a texture coordinate attribute to map that detail onto our mesh.
If however,
we are generating some procedural meshes like fractals,
it might make more sense to use a diffuse color attribute,
since all our detail will be generated at the same time as the mesh.

Keeping all this in mind,
the way this is implemented is by defining a single \dfn{struct} (code \ref{COD:VERT_STRUCT}).
This \dfn{struct} will contain fields for all supported attributes.
When writing the vertices to an OpenGL buffer,
unneeded attributes will be skipped and the \ac{vao} will only have the relevant attributes mapped to attribute pointers.

\LispCode[COD:VERT_STRUCT]{Vertex struct definition}{Vertex Struct Definition}{vertex-struct.lisp}{}{}{}

\subsection{Meshes}

A mesh is nothing more than a collection of vertices.
The vertices can be described in many ways,
but the most common is to do so by triangles.
When building a mesh up through triangles,
you are guaranteed that all faces will be flat,
since three distinct,
non-co-linear points always describe a plane.

In OpenGL there are many ways to represent meshes in its state.
The easiest way is to just have a \ac{vbo} with all vertices in the correct order.
This way quite inefficient though,
since any vertex that is shared among multiple faces will appear multiple times in the buffer,
taking up valuable space in memory.

\begin{equation}[EQ:VBO]{Example of a mesh described using only vertices in a VBO}
\boxed{VBO = \{ V_0, V_1, V_2, V_1, V_3, V_2, \dots \}}
\end{equation}

Another way that is very useful to represent arbitrary 3D shapes,
is by storing all unique vertices in the \ac{vbo},
and then using an \ac{ebo} to index into that buffer.

\begin{equation}[EQ:VBO_EBO]{Example of a mesh described using unique vertices and the indices}
  \boxed{
    \begin{matrix}
      VBO = \{ V_0, V_1, V_2, V_3, \dots \} \\
      EBO = \{ 0, 1, 2, 1, 3, 2, \dots \}
    \end{matrix}
  }
\end{equation}

\subsubsection{Mesh files}

One important thing to consider is how to store meshes on disk,
since their representation in memory can vary quite a bit.
Luckily there are many file formats to store mesh data.
One such format,
and the one I decided to implement a parser for,
is Wavefront OBJ.
The two main reasons are that it is a text based format,
making parsing it slightly easier than if it were binary,
and the fact that it is an open format,
making the specification very accessible.

The way the format works is that first it defines a set of vectors.
There are three types of vectors I implemented:
position vectors, texture coordinates, and normal vectors.
Then once all these vectors are defined,
all the surface faces get defined by indexing the vectors and combining them.
Since this format has support for face shapes other than triangles,
it is important to have mesh triangulation as part of the modeling workflow.

\subsubsection{Procedural mesh generation}

% Triangle mesh
% Triangle strip mesh (math)
% Hard edge normal calculation
% Interpolated normal calculation
% Vertex transformations based on functions

Sometimes it is desirable to render shapes that aren't modeled beforehand.
In this document I will refer to the creation of those shapes as procedural mesh generation.
In this project this process takes place in two steps,
first a base mesh is generated,
and then certain transformations are applied to every vertex in the mesh to reach a new shape.
I explain this process in more detail in section \ref{SEC:PROCEDURAL}
