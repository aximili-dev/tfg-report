% Talk about quicklisp and asdf
% Talk about emacs and sly

\label{SEC:DEV_ENV}

\subsection{Package management}

Like many programming languages,
Common Lisp does not have a native build and package management system.
I used two pieces of software to help with that.
The first is Quicklisp,
a library manager.
It is supported by many Common Lisp implementations,
and provides access to over 1500 libraries.\cite{quicklisp}
In order to actually specify the libraries needed though,
I used \ac{asdf}.
It is the ``\textit{de facto} build facility for Common Lisp''\cite{asdf}.
With it,
you can describe how a lisp project is organized,
and how it should be built and loaded.

Together,
Quicklisp and \ac{asdf} take care of downloading necessary dependencies,
loading their packages so that their functions can be used,
building and running the software that you write,
and finally they can compile and link a standalone binary that can be run its supported platforms.

\subsection{Integrated Development Environment}

As mentioned in section \ref{SEC:CL},
Common Lisp is developed against a \ac{repl}.
For this purpose,
I set up the Emacs editor to act as an \ac{ide}.
Emacs is well-suited for Common Lisp development,
since it is built with a different flavor of lisp,
Emacs Lisp.
Apart from that,
there exist several packages for Emacs that integrate the editor with the \ac{repl}.
I chose to install SLY,
because of its many useful debugging features.
I've included my full Emacs configuration in appendix \ref{APP:EMACS} for reference.

\subsubsection{SLY}

The helpfulness of SLY is easily understated,
so this section is dedicated to explaining some of its features in detail\cite{sly}.

\paragraph{Full-featured REPL}

The default \ac{repl} provided by most Lisp implementations is very bare.
SLY provides a full-featured \ac{cli}.
The additional functionality includes completion,
reverse i-search and bash-style keyboard navigation.

\paragraph{Debugger}

Whenever an error is encountered in a Lisp program,
execution will pause and a debugger will show up in the \ac{repl}.
SLY improves on this by letting the developer explore the full execution stack,
inspect and modify variables,
and evaluate arbitrary Lisp code before continuing or aborting execution.

\paragraph{Stickers}

Stickers are similar to breakpoints in other debuggers,
but they are set for arbitrary expressions.
When they are enabled,
SLY will pause execution after evaluating a marked expression,
allowing you to inspect the result without having to create intermediary variables.
