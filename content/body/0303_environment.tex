% Talk about quicklisp and asdf
% Talk about emacs and sly

\label{SEC:DEV_ENV}

\subsection{Package management}

Like many programming languages,
Common Lisp does not have a native build and package management system.
I used two pieces of software to help with that.
The first is Quicklisp,
a library manager.
It is supported by many Common Lisp implementations,
and provides access to over 1500 libraries.\cite{quicklisp}
In order to actually specify the libraries needed though,
I used \ac{asdf}.
It is the ``\textit{de facto} build facility for Common Lisp''\cite{asdf}.
With it,
you can describe how a lisp project is organized,
and how it should be built and loaded.

\subsection{Integrated Development Environment}

As mentioned in section \ref{SEC:CL},
Common Lisp is developed against a \ac{repl}.
For this purpose,
I set up the Emacs editor to act as an \ac{ide}.
Emacs is well-suited for Common Lisp development,
since it is build with a different flavor of lisp,
Emacs Lisp.
Apart from that,
there exist several packages for Emacs that integrate the editor with the \ac{repl}.
I chose to install SLY,
because of its many useful debugging features.
I've included my full Emacs configuration in appendix \ref{APP:EMACS} for reference.
