The main objective is to be capable of \Index{rendering} 3D scenes.
However, 2D scenes should also be possible.
The rendering will be hardware accelerated on platforms that support it.
In order to render something it will need a mesh and a material.
The \Index{renderer} selects a specific shader depending on the type of mesh and meterial.

\paragraph{Deferred shading}
\Subindex{shading}{Deferred shading}, or \Subindex{rendering}{deferred rendering} is a technique which optimizes lighting calculations,
allowing for larger scenes with more light sources.
The way this works is by rendering the scene in multiple passes.
In the initial stage we render each mesh individually,
calculating the position, normal, albedo and specular index for each pixel.
In the lighting pass,
we use this information to do our lighting calculations,
ensuring we only calculate each pixel's lighting once.\cite{deferred_shading}
