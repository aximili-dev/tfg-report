\index{Rendering}

A graphics library is a collection of abstractions,
which allow developers to make use the capabilities of rendering hardware,
such as graphics cards.
Many such libraries exist,
ranging from low level \ac{api}s that only offer rendering capabilities,
to higher level ones that provide additional functionality.
To provide one example of a high level library,
% TODO: cite
BGFX\cite{bgfx} is a ``cross-platform, graphics API agnostic rendering library''.
Graphics API agnostic means that it targets many low level graphics \ac{api}s.

Most of these libraries and \ac{api}s are implemented for many programming languages,
and when they aren't,
bindings are usually available.
In the case of Common Lisp however,
OpenGL is the only API available with a somewhat actively maintained library.
Some others exist,
and those will be touched upon in section \ref{SEC:SOTA}.

\subsection{OpenGL}

\index{OpenGL}

OpenGL\cite{khronos} is a cross-language,
cross-platform graphics API for rendering vector graphics.
By the nature of the specification,
it has support for 2D and 3D graphics,
but the library itself doesn't distinguish those.

In order to function,
an OpenGL context needs to be instantiated.
This context then maintains state,
which will instruct the graphics hardware how to operate.
To render a mesh for example,
the context needs to be manipulated into the corresponding state,
and then a rendering command can be executed.

\subsubsection{OpenGL example}

Working directly with this API is very cumbersome,
especially as the scale of a project grows.
As an example,
here follows a program for rendering a single triangle,
which could be considered the simplest possible OpenGL program.
The example is implemented using the \texttt{cl-opengl}\cite{clopengl} library used for the rest of this project.

Before the program can even start,
mesh data and shader source code needs to be imported.
In the case of this example it has been hard-coded (code \ref{COD:GLINTRO_1}).
A detailed listing of the shader source code can be found in codes \ref{COD:EXAMPLE_TRI_VS} and \ref{COD:EXAMPLE_TRI_FS}.

% Global variables
\LispCode[COD:GLINTRO_1]{Basic OpenGL example \textemdash\, Global variables for 3D mesh and shader code}{Mesh data and vertex source code}{gl-basic-example.lisp}{1}{7}{1}

To start off,
a window needs to be created and an OpenGL context initialized.
Furthermore,
many OpenGL objects need to be generated to contain our data and state.
In code \ref{COD:GLINTRO_2} the \ac{vbo} and \ac{ebo} will hold vertex data and vertex indices respectively.
Variables also need to be created to store references to the individual shaders and the shader program.

% glfwInit and glfwCreateWindow
\LispCode[COD:GLINTRO_2]{Basic OpenGL example \textemdash\, Initialize OpenGL and create objects}{Window creation and OpenGL object generation.}{gl-basic-example.lisp}{9}{27}{9}

Code \ref{COD:GLINTRO_3} takes care of compiling the individual shaders and linking them together in a shader program.
Error checking and handling is omitted in this example,
but would add additional code and complexity. Code \ref{COD:GLINTRO_4} sets up the buffers generated earlier to hold mesh data.
Furthermore,
the \ac{vao} is configured so that the vertex layout can be understood by the shaders.

% Shader compilation and linking
\LispCode[COD:GLINTRO_3]{Basic OpenGL example \textemdash\, Shader compilation and linking}{Shader compilation and linking}{gl-basic-example.lisp}{29}{39}{29}

% Setting up vao
\LispCode[COD:GLINTRO_4]{Basic OpenGL example \textemdash\, Vertex array spec}{Mesh setup.}{gl-basic-example.lisp}{41}{54}{41}

Now that all the data is set up,
and the OpenGL context has the correct state,
some actual rendering can be done.
In code \ref{COD:GLINTRO_5} the only actual function that instructs the \ac{gpu} to render anything,
is \texttt{gl:draw-elements} on line 65.
The rest of the code surrounding it is again needed to manipulate state.

Finally, the code in listing \ref{COD:GLINTRO_6} cleans up the OpenGL context.
The result of executing all this code can be seen in \figref{FIG:EXAMPLE_1}.
Hopefully this example shows why a graphics engine with proper abstractions is not only useful,
but an absolute necessity for even the most basic graphics programming.

% Render Loop
\LispCode[COD:GLINTRO_5]{Basic OpenGL example \textemdash\, Render loop}{The actual rendering code. Of these lines, the only part that actually instructs OpenGL to render is the function call on line 65. The other functions only take care of getting the OpenGL context in the proper state. The final two lines instruct that the OS to draw the rendered framebuffer the window.}{gl-basic-example.lisp}{56}{74}{56}

% Cleanup
\LispCode[COD:GLINTRO_6]{Basic OpenGL example \textemdash\, Cleanup}{These final lines free up the OpenGL objects that were created at the beginning of the program.}{gl-basic-example.lisp}{76}{79}{76}
